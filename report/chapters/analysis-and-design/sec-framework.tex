\section{Frameworks}
\label{sec:Frameworks}

The framework is an important part of any software development project.
It provides a foundation for the project and can help speed up the development process. 
This section goes in depth on what the different machine learning frameworks
are and what they can provide to this project.

\subsection{Machine Learning frameworks}

Choosing the correct machine learning frameworks is crucial when considering development,
training and deployment of an Artificial Intelligence (AI) model.
There are many frameworks available, but among the most widely used are TensorFlow
and PyTorch. They have both appeared dominant in their respective fields and are used
by some of the biggest tech companies in the world, like Google, Microsoft and Meta.\cite{PyTorchVsTensorFlow}

\subsubsection{TensorFlow}
TensorFlow was the first of the of the 2 previously mentioned frameworks to be released,
and was developed by Google. \cite{TensorFlow} It is widely adopted for
larger-scale machine learning applications and for production-grade AI systems,
because of its scalability and robustness. \cite{simplilearn}
Because of these features, TensorFlow has become a very versatile framework,
but it has a higher learning curve and its static computation graphs can hinder our development.

\subsubsection{PyTorch}
PyTorch was developed by Meta and it quickly gained popularity after its release due to its dynamic
computation graphs, better debugging capabilites and pythonic approach to development. \cite{PyTorch}

PyTorch might be limited in production readiness and have inferior performance compared to
TensorFlow when working on large-scale projects, but because of its strenghts and being faster
than TensorFlow on small-scale models it has made a huge leap in favorability in
scientific-research and small-scale projects. \cite{simplilearn}

\subsection{Design decision: ML Frameworks}
The selection of a machine learning framework is a key architectual decision, that
will have a significant impact on the models development, debugging, training efficiency,
and integration with other components. The two leading frameworks that were considered were
TensorFlow and PyTorch. 

\textbf{TensorFlow}, developed by Google, is widely adopted in the industry for building 
and deploying large-scale machine learning systems. It offers robust tools along with
a versatile framework that supports production-grade AI applications. Its static computation
graphs can improve performance in production scnearios and is generally favoured for 
large-scale applications. However, TensorFlow comes with a much steeper learning curve,
especially for beginners. Its static graph model can complicate debugging and 
limit flexibility during development often requiring additional abstraction layers.

On the other hand, \textbf{PyTorch}, developed by Meta, is known for its dynamic
computation grpahs, which allows for greater flexibility and ease of use. It adopts a
more pythonic coding style, making it more intuitive to developers already familiar with
Python. PyTorch also integrates well with tools such as Jupyter notebooks and Gymnasium,
which are useful for reinforcement learning tasks and interactive development. Additionally,
PyTorch has shown faster training times for smaller-scale models, which aligns well with
the scope and scale of this project.

Despite the limitations in production-readiness and scalability and the fact that the
Poke-env environment that we are using is made using TensorFlow, the decision was made
to use PyTorch for this project. The reasons for this was due to its flexibility, better
debugging support and smoother development experience and learning curve. The advantages
were particularly important given the iterative nature of designing and tuning a reinforcement learning
agent, where frequent model adjustments and state inspections are necessary. 

In conclusion, PyTorch provided a more accessible and developer-friendly framework for
implementing and experimenting with Deep Q-Networks, making it a suitable choice for this project.


