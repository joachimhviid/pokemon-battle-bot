\section{Frameworks}
\label{sec:Frameworks}

Frameworks is an important part of any software development project.
They provide a foundation for the project and can help speed up processes
during devleopment. This section goes in depth on what the different frameworks
are and can provide to this project.

\subsection{Machine Learning frameworks}

Machine learning frameworks is a crucial choice to make when considering development,
training and deployment of an Artificial Intelligence (AI) model.
There are many frameworks available, but among the most widely used are TensorFlow
and PyTorch. They have both appeared dominant in their repective fields and are used
by some of the biggest tech companies in the world, like Google, Microsoft and Meta.\cite{PyTorchVsTensorFlow}

\subsubsection{TensorFlow}
TensorFlow was the first of the of the 2 frameworks to be released,
and was developed by Google. \cite{TensorFlow} It is widely adopted for
larger-scale machine learning applications and for production grade AI systems,
because of its scalability and robustness. \cite{simplilearn}
Because of all these features TensorFlow has become a very versatile framework,
but has created a higher learning curve and its static computation graphs can hinder our development.

\subsubsection{PyTorch}
PyThorch was developed by Meta and after its release it quickly gained popularity due to its dynamic
computation graphs, better debugging capabilites and pythonic approach. \cite{PyTorch}

PyTorch might be limited in production readiness and have inferior performance compared to
TensorFlow when working on larger scale projects, but because of its strenghts and being faster
than TensorFlow on smaller-scale models it has made a huge leap in favorability in
scientific-research and smaller scale projects. \cite{simplilearn}

\subsection{Design decision: ML Frameworks}


