\section{Data source}
\label{sec:data-source}

In order to simulate a Pokemon battle, it is necessary to have the data needed for a Pokemon battle to take place.
The simulator needs data on Pokemon species and their stats and moves, as well as data on what the individual moves do.
It needs to know about Pokemon abilities and held items, and how they interact with eachother.

To this end it is important to find a good data source to work with. The following are some options to consider
for this purpose.

\subsection{Data mining}
Data mining is the process of discovering patterns in large data sets \cite{DataMining}. It is commonly used in video games to extract raw data 
from game files, such as assets (3D models, sprites, etc.) or hidden details of how certain functions interact in a game. This data has historically
been used for online documentation of game mechanics \cite{DataMiningPokemon} or even for creating fan-made modifications to the original game \cite{RenegadePlatinum}.

For the purpose of creating a battle simulator, it would be possible to "mine" the data directly from the game files. This is a lengthy process, however
and can be viewed as a grey area in terms of legality or ethicality. This is because it involves sideloading software on a physical console in order to
extract encrypted data from both the console and the game cartridge itself. Once in possesion of the keys and game data, it is then possible to extract
data by removing multiple layers of encryption and parsing Nintendo's proprietary file formats.

This method would be the most accurate data source available, but is also the most tedious option to use compared to other data sources.

\subsection{Kaggle}
\subsection{PokeAPI}
