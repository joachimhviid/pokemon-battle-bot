\section{Data source}
\label{sec:data-source}

In order to simulate a Pokemon battle, it is necessary to have the data needed for a Pokemon battle to take place.
The simulator needs data on Pokemon species and their stats and moves, as well as data on what the individual moves do.
It needs to know about Pokemon abilities and held items, and how they interact with eachother.

To this end it is important to find a good data source to work with. The following are some options to consider
for this purpose.

\subsection{Data mining}
Data mining is the process of discovering patterns in large data sets \cite{DataMining}. It is commonly used in video games to extract raw data 
from game files, such as assets (3D models, sprites, etc.) or hidden details of how certain functions interact in a game. This data has historically
been used for online documentation of game mechanics \cite{DataMiningPokemon} or even for creating fan-made modifications to the original game \cite{RenegadePlatinum}.

For the purpose of creating a battle simulator, it would be possible to "mine" the data directly from the game files. This is a lengthy process, however
and can be viewed as a grey area in terms of legality or ethicality. This is because it involves sideloading software on a physical console in order to
extract encrypted data from both the console and the game cartridge itself. Once in possesion of the keys and game data, it is then possible to extract
data by removing multiple layers of encryption and parsing Nintendo's proprietary file formats.

This method would be the most accurate data source available, but is also the most tedious option to use compared to other data sources.

\subsection{Kaggle}
Kaggle is a platform owned by Google, that is home to a large number of datasets and models used by data scientists \cite{WhatIsKaggle}.
The datasets are made by the community in order to allow data scientists to practice their craft without having to worry about gathering large amounts of
data themselves. 

Kaggle is home to over 400 Pokemon datasets \cite{PokemonKaggleDataSets} covering a vast spectrum of topics. For this project, data on Pokemon species,
moves and abilites is required. This data does exist on Kaggle, however not fully complete nor available in a single set. The most complete set \cite{PokemonDataSetWithStats} is 
more than 8 years old and as such does not meet the requirement of generation 9 mechanics described in the functional requirements (see table \ref{tab:functional-requirements}).
Additionally it would have to be supported by a secondary dataset that contains available Pokemon moves and abilites.

This data source is potentially viable for the project, but would require some extra work in order to organize multiple datasets into a cohesive source.

\subsection{PokeAPI}
PokeAPI is an open-source REST API featuring all available Pokemon data \cite{PokeAPI}. Its data is sourced by multiple different projects, most notably 
Veekun \cite{Veekun} a site with a collection of data mined info from the Pokemon games. PokeAPI even has some wrapper libraries that can be easily integrated
into the project \cite{PokeAPIWrapperLibs}. 

PokeAPI contains all the data needed for the project and is neatly organized in their API. It contains all the Pokemon species and their various forms, 
all of their abilites, moves and types, as well as all of the held items that can be used in a Pokemon battle. 

All of this makes PokeAPI a strong candidate for the projects data source, both in terms of data completeness, but also ease of use.


\subsection{Design Desicion: Data Source}
Selecting a reliable and appropriate data source was a diffucult step in the design
of this project, as the quality, structure and source of the data could possibly 
affect the performance of the system. Several option were considered each with their
own pros and cons.

\textbf{Data mining} was evaluated to be the most accurate data source, as it would
provide the most complete and up-to-date data directly from the game files. This method
however comes with some significant drawbacks. It requires a lot of technical skills and tooling
to extact and implement the data correctly, which introduces complecity and a high development overhead.
Moreover, data mining from a prprietary game is a grey area in terms of legality and raises ethical 
and legal concerns, making it less desirable for this project.

Another option was to use \textbf{Kaggle}, a popular platform known for its vast collection
of datasets and strong community support. Kaggle procides access to a wide range of Pokemon-related
datasets, which could potentially accelerate development. However, many datasets on Kaggle
may be outdated, inconsistent, incomplete or even AI-generated, leadning to concerns about data 
validity and reliability. Additionally, the large amount of datasets can make it difficult to 
identify which dataset is best suited for the project's requirements and specifications.

Ultimately, the decision made was to use \textbf{PokeAPI} as the data source for the project. PokeAPI
is a well-maintained and widely used public API that provides structured and comprehensice Pokemon data,
including species, movesets, abilities, items and alot more. It is easy to integrate into
a Python-based project, which is the language of choice for this project. The API is 
supported by a strong community, making it easier to troubleshoot issues and ensure long-term'
mainttainability. While reliance on an external API does introduce som risk of downtime and 
requreies an internet connection, these concers were considered manageable, within the scope of the project.


