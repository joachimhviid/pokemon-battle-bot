\section{Pokemon mechanics}
\label{sec:pokemon-mechanics}

The Pokemon games are home to a vast amount of mechanics. In order to fully understand
this project, it is important to have a basic understanding of the mechanics present in a Pokemon battle.

The following section aims to briefly explain these mechanics.

\subsection{Type chart}
A Pokemon will under normal circumstances have 1 or 2 types that represents its elemental alignment and similarly a Pokemon move will have 1 type.
Each type determines its effectiveness against other types. The effectivenesses are categorized as follows:
\begin{itemize}
  \item No effect (0x multiplier)
  \item Not very effective (0.5x multiplier)
  \item Effective (1x multiplier)
  \item Super effective (2x multiplier)
\end{itemize}
These multipliers are stacking, so if a Pokemon has 2 types that resists an incoming attack, the multiplier becomes $ 0.5*0.5=0.25 $.

Example: A fire type Pokemon, Charmander, is in a battle with the water type Pokemon, Squirtle.
The Squirtle uses the water type move Water Gun against the Charmander. As water is super effective against fire, the move 
Water Gun receives a 2x multiplier to its damage.

\subsection{Stats}
\subsubsection{IVs and EVs}
\subsubsection{Stat boosts}
\subsection{Abilites}
\subsection{Held items}
\subsection{Status conditions}
\subsubsection{Poison}
\subsubsection{Burn}
\subsubsection{Sleep}
\subsubsection{Paralysis}
\subsubsection{Freeze}
\subsubsection{Confusion}
\subsubsection{Flinch}
\subsubsection{Seed}
\subsubsection{Infatuation}
\subsubsection{Curse}
\subsubsection{Nightmare}
\subsection{Moves}
% Power
% Damage range
% OHKO
% Critical hits
% Physical/Special
% Accuracy
% Priority
% Targeting
\subsection{Weather}
\subsection{Field effects}
\subsection{Battles}
% Turn order
\subsubsection{Double battles}
\subsubsection{Special battles}
\subsection{Special mechanics}
\subsubsection{Mega Evolution}
\subsubsection{Z-moves}
\subsubsection{Dynamax and Gigantomax}
\subsubsection{Terastalization}
