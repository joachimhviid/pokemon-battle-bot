\section{Pokemon mechanics}
\label{sec:pokemon-mechanics}

The Pokemon games are home to a vast amount of mechanics. In order to fully understand
this project, it is important to have a basic understanding of the mechanics present in a Pokemon battle.

The following section aims to briefly explain these mechanics.

\subsection{Type chart}
A Pokemon will under normal circumstances have 1 or 2 types that represents its elemental alignment and similarly a Pokemon move will have 1 type.
Each type determines its effectiveness against other types. The effectivenesses are categorized as follows:
\begin{itemize}
  \item No effect (0x multiplier)
  \item Not very effective (0.5x multiplier)
  \item Effective (1x multiplier)
  \item Super effective (2x multiplier)
\end{itemize}
These multipliers are stacking, so if a Pokemon has 2 types that resists an incoming attack, the multiplier becomes $ 0.5*0.5=0.25 $.
\notebox{Example: A fire type Pokemon, Charmander, is in a battle with the water type Pokemon, Squirtle.
The Squirtle uses the water type move Water Gun against the Charmander. As water is super effective against fire, the move 
Water Gun receives a 2x multiplier to its damage.}

For the full list of type advantages and disadvantages, see appendix \ref{appendix:type-chart}. 

\subsection{Stats}
\subsubsection{IVs and EVs}
\subsubsection{Stat boosts}
\subsection{Abilites}
\subsection{Held items}
\subsection{Status conditions}
In pokemon there exists a variety of status conditions that fall into either of two categories: Volatile and Non-volatile. Non-volatile status conditions
lasts until a pokemon is healed, and volatile conditions only lasts for the duration of a battle. \cite{StatusCondition}
\begin{itemize}
  \item Poison (PSN): Is a condition non-volatile that deals 1/8 of the pokemon's health at the end of each turn and can only be avoided by a few abilities,
  moves or items.
  \begin{itemize}
    \item Badly Poisoned: Is another type of the non-volatile poison status condition that starts to deal 1/16 at the start of a turn, 
      but increases the damage taken by 1/16 every turn, so after 3 turns it deals 3/16 of a pokemons health. 
      This condition can be avoided by the same abilities, moves or items as the normal poison condition.
  \end{itemize}
  \item Burn (BRN): Is another non-volatile status condition that deals 1/16 from generation 1 and 7 onwards while also cutting the pokemons attack stat
    in half until the pokemon is heald. There are one ability and move that negates the attack drop from burn and turns it into an attack boost instead.
  \item Sleep (SLP): This condition causes a pokemon unable to use moves, except snore and sleep talk and 
    randomly lasts 1 to 3 turns in generation 5 and onwards. The first turn of sleep is always skipped.
  \item Paralysis (PAR): Is a non-volatile status condition that has a 25\% chance of preventing the pokemon from moving each turn and reducedes a pokemons
    speed 50\% from generation 7 onwards.
  \item Freeze (FRZ): This is a non-volatile condition that renders the pokemon unable to move unless, it thawes out by a 20\% chance or 
    by using a move to thaw it out.
  \item Confusion: Is a type of volatile status condition that has a 33\% chance of causing a pokemon to hurt itself in its confusion instead of executing
    a selected move.
  \item Flinch: The flinch status is volatile and prevents an oppossing pokemon from moving for the turn it was flinched. This can be done by moves like 
    fake out that has Priority +3, and rock slide that has a 30\% chance of flinching if you move before the oppossing pokemon.
  \item Others: There are a lot of other volatile status conditions that are not as common as the ones mentioned above.
  \begin{itemize}
    \item Grounded: Pokemon that are grounded looses their imunity to ground type moves.
    \item Nightmare: Only affects sleeping pokemon and deal 1/4 of their hp each turn they stay asleep.
    \item Curse: Deals 1/4 of the users hp each turn if used by a ghost type. If not lowers the users speed by 1 stage and increases 
      the users attack and defense by 1 stage.
  \end{itemize}
\end{itemize}
\subsection{Moves}
% Power
% Damage range
% OHKO
% Critical hits
% Physical/Special
% Accuracy
% Priority
% Targeting
\subsection{Weather}
\subsection{Field effects}
\subsection{Battles}
% Turn order
There exists a lot of different types of battles in the Pokemon games, the most common ones being Single battles and Double battles that have been present 
since generation 3. Each variation of battles builds upon the standard Single battle format. Single battles have 4 commands that can be done each turn:
\cite{PokemonBattles}
\begin{itemize}
  \item Battle: The player chooses a move to use against the opponent
  \item Pokemon: The player can switch to any pokemon in their party 
  \item Bag: The player can use items and berryies to heal their pokemon, this choice cannot be used in competitive battles.
  \item Run: The player can choose to run from a wild battle, but this choice cannot be used in trainer battles.
\end{itemize}
\subsubsection{Double battles}
Double battles are the main variation of the standard single battle format and is also the standard competitive format. In double battles 
each player sends out 2 pokemon at the start of the battle and can target any of the 4 pokemon on the field with their moves. Several moves change 
when used in a double battle, because they can be used to target both opponents, everyone except itself on the field or redirect moves.
This format gives a pokemon battle a lot more complexity and strategy than the single battle format, which is why it became the competitive standard.
Double battles also feature a mechanic that the single battle format doesnt have, a Dynamic turn order, which means that during a turn the pokemon
moving first can change during the turn. An example of having a dynamic turn order is with the move Tailwind tht doubles the speed of the users team for 
4 turns, and trick room that reverses the speed order of all pokemon on the field. \cite{PokemonBattles}

\subsubsection{Special battles}
As mentioned before there are a lot of different types of battles in the pokemon games, and some of them are special battles that have different rules
than the standard single and double battles. Some of the special battles are: \cite{PokemonBattles}
\begin{itemize}
  \item Sky Battle: Only flying type pokemon and pokemon that are levitating can participate in this type of battle.
  \item Triple battle: This type of battle is like playing through two double battles at once where only the middle pokemon is participating 
    in both double battles
  \item Rotation battle: Similiar to triple battle both players sends out 3 pokemons where you can switch between which of your pokemon is battling
    the opponents pokemon, though a key difference is that you dont hit multiple pokemon with moves like rock slide or earthquake. 
  \item Inverse Battle: During an inverse battle the type chart will be reversed.
  \item Battle Royal: A 4-way free for all battle where the last pokemon standing wins.
  \item Max Raid/Tera Raid battle: They are both kind of like a boss battle with a wild pokemon that use the 
  dynamax or gigantomax or Terastalization mechanic.
\end{itemize}

\subsection{Special mechanics}
Pokemon has begun to make flagship mechanics that are unique to one or two generation of pokemon games.
Currently they have four of these mechanics: Mega Evolution, Z-moves, Dynamax and Gigantomax and Terastalization.
\subsubsection{Mega Evolution}
Mega evolution was the first big flagship mechanic and was introduced in generation 6. It allows a pokemon to evolve mid-battle and gain a 
power boost, through a change in ability, typing or just pure stat change. This mechanic can only be used once per battle, but lasts for the whole battle
or until the pokemon is defeated and takes up the item slot of the pokemon.\cite{MegaEvolution}
\subsubsection{Z-moves}
Z-moves was a special type of move introduced in generation 7 and can only be used on per battle by a pokemon that holds a Z-crystal. if a player chooses to 
use a Z-move the player can choose any of the pokemons original list of moves which are compatible with the Z-crystal its holding and unleash a more powerful
version of that move. There exists a Z-crystal for each type and some special Z-crystals that can be used on their specific pokemon. \cite{Zmoves}
\subsubsection{Dynamax and Gigantomax}
Dynamax and Gigantomax increases a pokemons size drastically, changes their moves and increases their max hp by 50\% for 3 turns. This is again a one time
use per battle but lasts for 3 turns or until the pokemon is defeated. Gigantomax is a special form of Dynamax that only a few pokemon can use and 
changes the pokemons signature move into a G-max move that has a special effect on top of the damage boost it gets from being a Dynamax move.
This mechanic has as of now only been available during generation 8. \cite{Dynamax}
\subsubsection{Terastalization}
Terastalizing a pokemon means giving it one of the 19 tera types that can work either offensivly or defensivly. Pokemon remain Terastalized so long 
as they are alive and the battle is ongoing. Terastalized pokemon cannot revert back to their orignial typing, but still gain the STAB effects 
from their original types as well as the Tera type. \cite{TeraType}