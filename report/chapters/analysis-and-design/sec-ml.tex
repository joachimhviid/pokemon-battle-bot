\section{Machine Learning}
\label{sec:Machine Learning}

Machine learning is a subset of artificial intelligence 
that focuses on the development of a computer program that learns from data.
In machine learning there are three approaches: supervised, unsupervised, and reinforcement learning.
Each of these approaches has its own strengths and weaknesses, but all of them need a machine learning algorithm.

A machine learning algorithm is a set of rules or mathematical 
procedures that the system follows to learn from a dataset 
and make predictions or decisions based on the data.

After applying the algorithm a dataset, you will have what is called a Machine learning Model.
A model is a specific instance of a trained system that has learned from the data and can make predictions or classifications.
\cite{ML-Models}
\textbf{Analogy:} A machine learning model is like a fully cooked meal. 
The recipe (algorithm) was used to prepare it, but now it is a ready-to-use product.



\subsection{Supervised Learning}
The supervised learning approach is the more common approoach of the 3 approaches.
This is due to its ability to predict a wide range om problems accuratly, 
however its effectiveness is dependent on the quality of the training data.
Supervised learning uses labled dataset to train the model, 
which means that from the input data we expect the correct output data as well.
\cite{GoogleCloud-SL}

Some examples of supervised learning are spam email classifiers, 
so wheter an email is spam or not and weather prediction models.

\subsection{Unsupervised Learning}
Unsupervised learning is a type of machine learning 
that learns from data without human supervision. Unlike supervised learning,
unsupervised machine learning models are given unlabled data and allowed to discover patterns 
and relationships without any explicit guidance or instructions. \cite{GoogleCloud-UL}


\subsection{Reinforcement Learning}
Reinforcement learning is the third type of machine learning.
Unlike the previous branches of machine learning,
reinforcement learning relies on a datasets with predefined answers, it learns by experience. 
In reinforcement learning, an agent learns to achieve a goal in 
an uncertain and potentially complex evironment/s py recieveing feedback 
through rewards or penalties. \cite{RL-GeeksForGeeks}
\newline
The key concepts of reinforcement learning: 
\begin{itemize}
    \item Agent: The learner/decision maker
    \item Environment: Everything the agent interacts with.
    \item State: A specific situation the agent finds itself in. 
    \item Actions: What the agent can do
    \item Rewards: Feedback from the environment
\end{itemize}

 


