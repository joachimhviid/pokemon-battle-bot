\section{Choice of language}
\label{sec:choice-of-language}

When starting a large project like a battle simulator it is important to consider your options when it comes to choosing a technology to build with.
For a technical project, the initial building block is the programming language. While it is possible to do almost anything with any 
programming language, there are some that are more suited to certain tasks than others. For the battle simulator, a tool that is well 
suited for machine learning and handling large sets of data is desired.

\subsection{Python}
The first language that comes to mind, when considering machine learning capabilities, is Python \cite{PythonForMachineLearning}.
Python in general is known for its simplicity, which makes it easy to use in comparison to other languages. This ease of use will make it
easier to focus on what is important for this project: the machine learning.

Python is also home to a large ecosystem of libraries made specifically for the purpose of machine learning. Examples of these libraries are
TensorFlow, PyTorch and Pandas.


\subsection{C++}
C++ is another language to consider. It is a statically typed, compiled language, that is commonly used for high-performance systems, such as
game engines or operating systems \cite{C++}. Pokemon, the subject of this project's learnings, is also speculated to have been made using C++ in recent years \cite{PokemonProgrammingLanguageForumPost}\cite{NintendoDataLeak}\cite{ChatGPTPokemonProgrammingLanguage}.
C++ has a steep learning curve, but can produce more efficient applications compared to other languages. 
A part of C++'s complexity comes from manually controlling memory, thus providing more reliable systems if done correctly. 

When it comes to machine learning, the improved performance can make it possible to work with even more complex data and scenarios compared 
to Python \cite{C++VsPythonML}, as well as potentially reducing the time it takes to train a model. 

\subsection{JavaScript}
\subsection{Unity with C\#}
