\section{Choice of language}
\label{sec:choice-of-language}

When starting a large project like a battle simulator it is important to consider your options when it comes to choosing a technology to build with.
For a technical project, the initial building block is the programming language. While it is possible to do almost anything with any 
programming language, there are some that are more suited to certain tasks than others. For the battle simulator, a tool that is well 
suited for machine learning and handling large sets of data is desired.

\subsection{Python}
The first language that comes to mind, when considering machine learning capabilities, is Python \cite{PythonForMachineLearning}.
Python in general is known for its simplicity, which makes it easy to use in comparison to other languages. This ease of use will make it
easier to focus on what is important for this project: the machine learning.

Python is also home to a large ecosystem of libraries made specifically for the purpose of machine learning. Examples of these libraries are
TensorFlow, PyTorch and Pandas.

However, Python is an interpreted language with slow runtime performance compared to compiled languages such as C++. 
If performance becomes an issue it could be desirable to use a more performant language.


\subsection{C++}
C++ is another language to consider. It is a statically typed, compiled language, that is commonly used for high-performance systems, such as
game engines or operating systems \cite{C++}. Pokemon, the subject of this project's learnings, is also speculated to have been made using C++ in recent years \cite{PokemonProgrammingLanguageForumPost}\cite{NintendoDataLeak}\cite{ChatGPTPokemonProgrammingLanguage}.
C++ has a steep learning curve, but can produce more efficient applications compared to other languages. 
A part of C++'s complexity comes from manually controlling memory, thus providing more reliable systems if done correctly. 

When it comes to machine learning, the improved performance can make it possible to work with even more complex data and scenarios compared 
to Python \cite{C++VsPythonML}, as well as potentially reducing the time it takes to train a model. However, C++ does not come with a lot of libraries
for machine learning, so choosing C++ would involve creating a lot of tools before being able to start properly working on data.

\subsection{JavaScript}
\begin{quote}
    Atwood's Law: any application that can be written in JavaScript, will eventually be written in JavaScript.
    
    -- Jeff Atwood \cite{AtwoodsLaw}
\end{quote}

JavaScript is the main language used by the browser and has traditionally not been used for machine learning. In spite of this, Atwood's Law \cite{AtwoodsLaw} dictates
that machine learning in the browser was inevitable, and has in recent times been proven correct. JavaScript has received a number of libraries
for machine learning over the years, such as brain.js \cite{BrainJS}, ml5.js \cite{ML5JS} and even a JavaScript version of TensorFlow \cite{TensorFlowJS}.

Using JavaScript could be beneficial as it is easy to use, like Python, but runs on any platform that has a browser. JavaScript loses its edge over Python
when it comes to performance. While both languages are interpreted, JavaScript runs in the browser and as such only has access to WebGL and WebGPU. 
These APIs are not supported on all devices and are generally less efficient than the low-level compiled C libraries used in Python \cite{WebGPUvsCUDA}.
There is also a significant performance overhead because of these APIs that can slow down model execution \cite{DeepLearningInBrowsers}. Therefore JavaScript
might not be suitable for this project.

In F5 (see table \ref{tab:functional-requirements}) a GUI is described such that a user might be able to visualize the actions of an agent. JavaScript would be an 
excellent language choice for that purpose. A WebSocket connection could be made between the server running our model and the browser client displaying
the decisions made by the model.

\subsection{Unity with C\#}
Unity is a popular game engine used to create both 3D and 2D games and simulations using C\# \cite{Unity}. It can be used to create environments suitable
for training agents using various methods such as reinforcement learning or imitation learning \cite{UnityMLAgents}. 

Compared to Python, Unity has a steep learning curve,
due to the required knowledge in C\# and Unity's domain specific constructs based on game development concepts. Unity has excellent support for 3D environments and controls,
something that Python might struggle with out of the box. However, a Pokemon battle simulator is not bound to a physics enabled 3D environment and as such
Unity might be overly complex for the purposes of this project.
Unity is also lacking in certain traditional machine learning tasks such as data analysis, making further learnings more difficult to attain.  

\subsection{Choice of language}

\begin{table}[H]      
    \begin{tabular}{| m{3.5cm} | m{5cm} | m{5cm} |}
          \hline
          \textbf{Type} & \textbf{Pros} & \textbf{Cons} \\ 
          \hline
          \textbf{Python} & 
          \begin{itemize}
                \item 
          \end{itemize} & 
          \begin{itemize}
                \item 
          \end{itemize} \\ 
          \hline
          \textbf{C++} & 
          \begin{itemize}
                \item 
          \end{itemize} & 
          \begin{itemize}
                \item 
          \end{itemize} \\ 
          \hline
          \textbf{C\#} & 
          \begin{itemize}
                \item 
          \end{itemize} & 
          \begin{itemize}
                \item 
          \end{itemize} \\ 
          \hline
          \textbf{JavaScript} & 
          \begin{itemize}
                \item 
          \end{itemize} & 
          \begin{itemize}
                \item 
          \end{itemize} \\ 
          \hline
    \end{tabular}
    \caption{Table over our choice of programming language}
    \label{tab:choice-of-language}  
\end{table}
