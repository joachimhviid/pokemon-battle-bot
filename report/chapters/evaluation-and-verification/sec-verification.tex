\section{Verification}
\label{sec:verification}

This chapter evaluates the performance and learning progression og the Deep Q-network 
(DQN)-based pokemon battle agent over two iteration. The objective of this analysis 
is to validate the effectiveness of the chosen algorithm and wether we have achieved
the desired results based on our reqirements. Performance is assessed through empirical
training results, including average rewards and win rates.  

\subsection{Methodology}
Over the 2 iterations, the verification process was conducted using the same hyperparameters
but using different formats of the pokemon battle environment.
The first iteration used a signle battle format with randomly selected pokemon each
episode, while the second iteration was trained using the VGC 2025 regulation G format.
Each version of the agent, was evaluated over a series of training episodes, and data 
was colledted for:
\begin{itemize}
    \item Average reward per interval of 10 episodes
    \item Win rate per interval of 10 episodes.
    \item Training loss (rolling averages)
\end{itemize}
The figures below illustrate the training process over time for:
\begin{itemize}
    \item Iteration 1 - Baseline model trained over 1000 episodes \ref{fig:iteration-1-graphs}
    \item Iteration 2 - Enhanced model trained over 30000 episodes \ref{fig:iteration-2-graphs} 
\end{itemize}


\subsection{Results and analysis}
\textbf{Iteration 1 - Baseline model}
The first iteration exhibits very fluctuating performance, with relativly shallow 
state representation. Although the agent is capable of learning some degree of a control
policy, its reward curve is very erratic and lacks consistency.
\begin{itemize}
    \item The average reward fluctuates significantly, generally raning between -10 and +5
    indicating the agent struggles to consistently achieve positive outcomes.
    \item Our Agents win rate hovers around 0.4 to 0.6, which isnt generally bad, its almost a 50\% win rate
    and shows that it performs marginally better with random teams, but still with no clear upward trend.
    \item Training loss, remains flat indicating that something might be wrong with logging or infrequent updates.
\end{itemize}
This indicates that while the agent is learnig, its not effectivly generalizing 
information across different battle scenarios, largely due to shallow state information.

\textbf{Iteration 2 - Enhanced model}
In the second iteration, the agent is restructed to suppoort:
\begin{itemize}
    \item A more complex battle environment.
    \item More modular hyperparameters, allwing for more fine-tunning.
    \item A bigger Q-network, to account for the increased complexity.
\end{itemize}
This means the improvements to the second iteration have resulted in a smoother reward
distribution, altough still noisy, which is typical for reinforcement learning. The variance
is now centered closer to 0, and positive spikes are more frequent. The win rate 
has become more stable, despite some episode by episode variation 
and is now more consistently above 0.5 with several peaks reacing about 0.8 or 1.0.
Scalablity has also improved significantly. the agent is able to train over 30000 episodes
compared to the just 1000 episodes in the first iteration, without crashing or instability.

Notably, while the average reward trend is still noisy, the longer training horizon
and improved reward shaping, has allowed the agent to learn more stable strategies, over time.

\subsection{Discussion}
The improvements between iterations, can be attributed to several factors:
\begin{itemize}
    \item Expanded state space: Going from a state space of 92 features in the first iteration
    to 128 features in the second iteration, allowed the agent to capture more complex information
    about the environment, leading to better decision making.
    \item Replay buffer: Iteration 2's structured implementation of the replay buffer, improved 
    learning consistency and sample efficiency.
    \item Reward design: By incorporating move intermediate rewards (e.g., for fainting enemies or preserving HP),
    the agent receives more meaningful feedback per action, leading to smoother learning curves.
    \item Network Depth: A deeper DQN enabled our model to better capture complex patterns in the data.
\end{itemize}