\section{Process evaluation}
\label{sec:process-evaluation}

This project was made using agile principles and was split into 3 overall phases. Each phase corresponds to a section of
this report and had approximately 1 month of time allocated per phase. The phases were Project Planning 
(see chapters \ref{chap:introduction}, \ref{chap:requirements} and \ref{chap:analysis-and-design}), Development 
(see chapter \ref{chap:implementation}) and Evaluation and Conclusion (see chapters \ref{chap:evaluation-and-verification} 
and \ref{chap:conclusion}). At the end of each phase, the progress was presented to the supervisors in order to get some 
feedback and pointers for the work that needed to be focused on in the next phases. The goal of the phases were to mark 
important milestones in the project. Since the project was agile, the milestones did not prevent us from working on 
previous phase content, but they helped maintain focus on what was needed to be created at any given time.

The agile tools used by the project included, multiple weekly meetings discussing our progress, bi-weekly retrospective 
meetings and the use of Kanban boards to track tasks. The weekly meetings consisted of a mix of both online and in-person
meetings, to both discuss the project and do some collaboratory work like pair-programming or report writing. For more 
general communication, a Discord server was used. Here we also shared notes and sources we could use for the report and 
project.

Overall, these tools provided a strong foundation to our work process. Initially we had planned the phases to be 1 month
each, but as we progressed it became apparent that some phases needed more time than others. For instance, the
Development phase ended up taking closer to 2 months due to both technical difficulties during implementation, but also
due to assignments from different courses overlapping with the initial phase schedule.
The bi-weekly supervisor meetings and monthly presentations provided us with great feedback to work with in the initial
phases, but later in the process they were largely discontinued so that more time could be spent on building the project.
The weekly meetings with pair-programming also helped immensely early in the Development phase, but due to working different
aspects of the implementation these pair-programming sessions eventually evolved into debugging sessions, where we would
sit down and try to fix a technical problem together.
The Kanban board on Github helped maintain an overview of the tasks that needed to be done. It was split into Backlog, In Progress,
Review and Done. While the Kanban board was nice to have, we did not feel it provided too much value to the overall project.
This was likely due to the defined tasks being too broad or vague to definitively say whether a task was finished or not.
As a result, some tasks were accompanied by comments showing a long checklist of subtasks needed to complete the task. 
To borrow some terminology from SCRUM, some of the stories were in fact epics in disguise. Additionally, since this project
team consisted of two developers with very pre-defined tasks, the Kanban board didn't necessarily need to be checked as 
often, nor did it need to be as detailed as if the project team was larger.
To improve the value of the Kanban board, more time should have been committed to defining the units of work for each
developer. The tasks should have a clear start and end point, ideally with some kind of deliverable. Making the tasks more
granular this way could even have revealed some features to be easier to implement than initially anticipated and vice versa.
As a result, our time could be spent more efficiently and provide a better end product.
