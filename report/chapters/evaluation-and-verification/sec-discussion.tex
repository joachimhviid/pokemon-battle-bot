\section{Discussion}
\label{sec:discussion}

To round out this chapter, we will discuss the outcomes of this project. We will look back at what has been done
and reflect on how it compares with the initial problem statement. Then we will reflect on what we would do if we had more time.

\subsection{Did we solve the initial problem?}
Looking back to the problem described in section \ref{sec:problem-description}, it is stated that: 
\begin{quote}
  \itshape{We want to find out if it's possible to improve as a human player by analysing how a trained agent plays.}
\end{quote}
We created a custom environment and agent that was succesfully trained across thousands of episodes. After each training 
loop the agent became better and better. At the agents current level, it is unlikely to consistently beat a human opponent,
but given enough time and training, this agent will eventually become a stronger player, that we as humans can learn from.
Therefore, that part of the problem is considered solved.

An additional problem was described as:
\begin{quote}
  \itshape{We want to find out if it is possible to calculate the odds of winning a battle ahead of time given a predefined set of Pokemon.}
\end{quote}
This part was not fully implemented, but all of the necessary groundwork was created. The remaining work involved in solving
this part of the problem would be creating a command line interface where two files with team data could be specified. 
Then, the pre-trained AI model would be used to run 1.000-10.000 episodes depending on the desired accuracy. The amount of wins
and losses experienced during the episodes would then be returned to the user, who is then able to use this info to determine 
the odds of winning a given matchup. Additionally, a sample log could then be used to recreate the winning episodes.

\subsection{What would we do in the future?}
A lot of time was put into the creation of this project, however there is always room for improvements. If this project was to
be expanded upon, the first improvement would simply be to implement more mechanics. Going into this project, it was clear
that there would not be enough time to implement every single move, ability, held item and mechanics to the environment. 
For that reason, these mechanics were either completely omitted or only partially implemented in the current iteration, but
given more time, they must be fully implemented. During the implementation of the custom environment it dawned on us, just 
how much work there was in adding some of these mechanics, so this endeavour would also require a major refactoring in how
data is handled. This refactor would possibly look similar to how Pokemon Showdown \cite{PokemonShowdownSource} handles 
metadata of moves and abilities, by having multiple flags and events to listen to and extend.

Another thing that could be added with more time would be a graphical user interface (GUI). This would greatly increase 
useability for the users that do not know how to use the command line. The GUI could also include a builtin team builder 
instead of relying on a file input. In the far future, the GUI could be used as an overlay for the games to provide the 
user with move suggestions directly in the client.
