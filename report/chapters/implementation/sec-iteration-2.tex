\section{Iteration 2}
\label{sec:Iteration-2-Agent-Environment}
The second iteration of implementation focused on restructuring the agent for improved 
modularity, better representation of game states and scaleability. This version of the agent still
uses the Poke-env library, but the agent is now training in a double battle environment. The
agent is now capable of handling multiple Pokemon and can make decisions based on the state of 
both opponent pokemon and allied pokemon. The agent also uses PyTorches own replay buffer for
training, which is a significant improvement over the previous custom version.

\subsection{Agent}
Our agent is now capable of accepting pre-made teams so long a team alligns with the required battle
format. The agents hyperparameters are now also much more modular, allowing for easy tuning and
we have moved the training loop inside the agent class, so that the asynchronous methods only
sends and accept challenges from the agent in the environment. The state size is now also 
calculated based on the number of pokemon in the team, and the number of pokemon in the opponents 
and their type. This means we have a lot more options for the agent to choose from, and learn 
the pokemons types. This also means that the Q-function approximator is now a lot more complex,
because the agent is now capable of learning from a much larger state space. 


\subsection{Reward Function}
This new reward function is an improvement over the previous one, as it now takes into accounts
the HP of the all pokemon on the field, and the amount left after the battle. The reward function
is an important part of the training process, as it not only gives us visual data to measure, but
also allows the agent to learn and adapt to the environment. This will give us a better understanding
of how well the agent is learning and its performance. In the previous iteration, the reward
function only gave a reward based on winning and losing, and taking an average of og the number of
times won and lost gave us a flat curve, because there was no real variation in the data.
The new reward function could also be further improved, by adding a reward for a number of turns
or increasing the reward for a winning streak. This would allow the agent to make a repoducable output
quicker, and would also allow us to measure the agents performance over time.
The reward function is as follows:


\subsection{Replay Buffer} % Explain 
There are two parts to the replay buffer, the first is storing observed experience tuples in a replay 
memory, and the second is to push small batches of experience tuples randomly, to help train the agent. 
The replay buffer is a key part of the training process, as it allows the agent to learn from its own 
experience, and allows us to update our target Q-network dynamically during training. The experience
tuples stored are as follows: 
\begin{itemize}
    \item \textbf{State}
    \item \textbf{Action}
    \item \textbf{Reward}
    \item \textbf{Next State}
    \item \textbf{Done}
\end{itemize}



