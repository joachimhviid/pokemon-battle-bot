\chapter{Requirements}
\label{chap:requirements}

In any development process, a clear and well-defined set of requirements is critical
to ensuring the success of the project. Requirements serve as a foundation for design, implementation,
verification, and validation activities throughout the software development lifecycle.
Without a structured and comprehensive understanding of what the system should do and how it 
should perform, there is risk of misalignment, scope creep, technical debt and ultimately, 
project failure. This chapter outlines the functional and non-functional requirements for the 
system. Functional requirements describe specific behaviors or functions in the system, detailing
core capabilities that directly fulfill the intended purpose of the software.
Non-functional requirements, on the other hand, define quality attributes that the system
should exhibit. These include performance, scalability and usability among others. While
non-functional requirements may not be directly observable through the system's interface, they
still have a significant impact on the user experience and overall system effectiveness.
To support effective planning and delivery, the requirements in this chapter are categorized 
using the MoSCoW method:
\begin{itemize}
    \item Must Have - Essential features that without the system cannot function.
    \item Should Have - Important features that add significant value but are not critical 
    for a minimum viable product (MVP).
    \item Could Have - Desirable features that may be included if time and resources allow for it.
    \item Won't Have - Requirements that explicitly are deferred for future development or out of scope.
\end{itemize}

\section{Functional Requirements}
\label{sec:functional-requirements}

\newcounter{requirement}
\newcounter{subrequirement}
\newcommand\rownumber{\arabic{requirement}}
\newcommand\subrownumber[1]{%
  \stepcounter{subrequirement}%
  \arabic{requirement}.\arabic{subrequirement}%
}
\newcommand\resetrownumbers{\setcounter{requirement}{0}\setcounter{subrequirement}{0}}

\begin{xltabular}{\textwidth}{
  >{\raggedright\arraybackslash}m{1cm} |
  >{\raggedright\arraybackslash}X |
  >{\raggedright\arraybackslash}X |
  >{\centering\arraybackslash}m{2cm}
  }
  \textbf{ID}                                              & \textbf{Title}          & \textbf{Description}                                                                             & \textbf{MoSCoW} \\\hline
  \stepcounter{requirement}F\rownumber                     & Battle Environment      & Turn-based battle simulator that can take inputs from an agent.                                  & M               \\\hline
  F\subrownumber{\rownumber}                               & Gen 9 Mechanics         & The environment will be based on the combat mechanics currently available in Generation 9 games. & M               \\\hline
  F\subrownumber{\rownumber}                               & Game Specific Mechanics & Game specific mechanics such as Mega Evolution, Z-moves, Terastalization and Dynamaxing.         & W               \\\hline
  F\subrownumber{\rownumber}                               & Uncommon status conditions & Certain status conditions are very rare and don't regularly affect the outcome of a battle.         & C               \\\hline
  F\subrownumber{\rownumber}                               & Uncommon unique moves & Certain unique moves that require custom implementation and don't regularly appear in the games, such as moves exclusive to a single Pokemon species.         & C               \\\hline
  F\subrownumber{\rownumber}                               & Held Items              & The environment supports a subset of held items.                                                 & M               \\\hline
  F\subrownumber{\rownumber}                               & Single Battles          & The environment must support single battles with 1v1 combat.                                     & M               \\\hline
  F\subrownumber{\rownumber}\setcounter{subrequirement}{0} & Double Battles          & The environment must support double battles with 2v2 combat.                                     & S               \\\hline
  \stepcounter{requirement}F\rownumber                     & Trainable Agent         & An agent that can be trained in the battle environment.                                          & M               \\\hline
  \stepcounter{requirement}F\rownumber                     & Configurable Teams      & The sets of Pokemon used in the environment can be changed.                                      & S               \\\hline
  \stepcounter{requirement}F\rownumber                     & Team Builder            & Build a team using a GUI for the agent(s) to use.                                                & C               \\\hline
  \stepcounter{requirement}F\rownumber                     & GUI                     & A GUI for viewing the agents actions and watching combat logs in real time.                      & C               \\\hline
  \caption{Functional Requirements}
  \label{tab:functional-requirements}
\end{xltabular}

\section{Non-functional Requirements}
\label{sec:non-functional-requirements}

\resetrownumbers

\begin{xltabular}{\textwidth}{|
  >{\raggedright\arraybackslash}m{1cm} |
  >{\raggedright\arraybackslash}X |
  >{\raggedright\arraybackslash}X |
  >{\centering\arraybackslash}m{2cm}
  |}  % first column with fixed width, others flexible
  \hline
  \textbf{ID}                           & \textbf{Title}       & \textbf{Description}                                                                                          & \textbf{MoSCoW} \\\hline
  \stepcounter{requirement}NF\rownumber & Reproduceable Output & Given an agent and an input the outcome should be reproducable, unless a high variance occurs                 & S               \\\hline
  \stepcounter{requirement}NF\rownumber & Portable             & The agent should function on both Windows and MacOS.                                                          & M               \\\hline
  \stepcounter{requirement}NF\rownumber & Runtime              & The simulator should be able to run 100-500 iterations within 10 seconds.                                     & M               \\\hline
  \stepcounter{requirement}NF\rownumber & Training time        & Each iteration of the agent should be trained within 6 hours.                                                 & S               \\\hline
  \stepcounter{requirement}NF\rownumber & User Experience (UX) & User evaluation test to make sure our GUI gives users a good experience when watching.                        & C               \\\hline
  \caption{Non-functional Requirements}
  \label{tab:non-functional-requirements}
\end{xltabular}

